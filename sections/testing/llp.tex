The algorithms related to parsing and constructions of $LLP(q, k)$ are tested. Some precomputed examples from the LLP paper \cite{Vagner2007} are used to test that each algorithm arrives at the correct result. This is because there is not really any easy way of implementing the PSLS definition. The second problem is computing the LLP Item collection by hand would be quite a daunting task. Therefore, the LLP collection in example 11. \cite[14]{Vagner2007} is used to assert correctness and the PSLS table in example 12. \cite[14]{Vagner2007}. These two single tests helps in directing the correctness of the LLP collection and PSLS table generation but does not help with asserting the correctness of the parser.

To assert the validity of the parsers generated grammar 11. \cite[14]{Vagner2007} is once again used. It has epsilon productions and nonterminals in a row, so it still seems like a good choice for testing the parser on a smaller scale. To do this parser $LLP(q, k)$ with $q, k \in {1, 2, 3}$ are constructed resulting in nine parsers. These parsers are tested by constructing all leftmost derivable strings of a given length and testing if the parsers can parse all these strings. The parsers are all tested on all strings of a given length which are not leftmost derivable strings and should fail on all these strings.