Unit testing with some property based testing are used to ensure the first and follow sets are computed correctly by Algorithm \ref{algorithm:first} and \ref{algorithm:follow}. The structure of the first and follow set tests is some small hard coded tests and tests which assert the first and follow set definition is fulfilled. The small tests are meant to be easy to comprehend and help guide the programmer with large errors. While the first and follow set tests are used to assert the correctness.

\subsubsection{Unit tests}
The unit tests used to assert the correctness of the first and follow sets are done by comparing the computed sets with precomputed sets of some grammars. The existing results were taken from \cite[58, 62, 63, 65]{Mogensen} with some small modifications to the results where $\epsilon$ is included in the first and follow sets due to the LLP paper definition \cite[5]{Vagner2007}. These grammars are complex enough to cover different possible properties of grammars which can influence the sets. Such as nonterminals followed by each other and nullable nonterminals.

\subsubsection{Property based like testing}
These test are like property based tests which tests for some property, but they tested on random input. The property tested for is if the functions can reconstruct the LLP paper definition of first and follow \cite[5]{Vagner2007}.
\begin{align*}
    \FIRST_k(\alpha) &= \{x : x \in T^* : \alpha \Rightarrow^* x \beta \land |x| = k\} \\ 
    \, &\cup \{x : x \in T^* : \alpha \Rightarrow^* x \land |x| \leq k\} \\
    \FOLLOW_k(A) &= \{x : x \in T^* : S \Rightarrow^* \alpha A \beta \land x \in \FIRST_k(\beta)\}
\end{align*}
This can be done by implementing the naive implementation described later. For the property based tests $k \in \{1, 2, 3, 4\}$ is used for two grammars \cite[62, 63]{Mogensen}. These tests could have been more extensive with more $k$ values or random grammars. This was never done because the naive first and follow implementations becomes extremely slow for larger $k$.

\paragraph{\texorpdfstring{$\FIRST_k$}{TEXT} testing}
The $\FIRST_k$ definition can be naively implemented by doing a breadth first search on derivable strings for each nonterminal of a grammar. If the first sets of all nonterminals are reproducible by Algorithm \ref{algorithm:first} then the $\FIRST_k$ function is computed correctly for the given grammar. The reason for this is you are able to compute any $\FIRST_k$ of a string in a grammar if you know $\FIRST_k$ of any nonterminal.

\paragraph{\texorpdfstring{$\FOLLOW_k$}{TEXT} testing}
The $\FOLLOW_k$ definition can also be naively implemented by doing a bread\-th first search on derivable strings from the starting nonterminal. Every time a nonterminal $A$ occurs in the derivable string the $\FIRST_k$ function is computed of the trailing symbols $\beta$. These sets are then used to construct the $\FOLLOW_k(A)$ set. If all $\FOLLOW_k(A)$ are reproducible by Algorithm \ref{algorithm:follow} then the $\FOLLOW_k$ function is computed correctly for the given grammar.

