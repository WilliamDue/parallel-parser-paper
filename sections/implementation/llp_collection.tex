When constructing the LLP collection of items sets Algorithm 8 \cite[13]{Vagner2007} is used. The actual Haskell implementation of algorithm 8 should match the implementation, but some implementation details would be needed to be explained. In the original Algorithm 8 it was said that. 
\begin{center}
    step 3. (a): $\gamma$ is the shortest prefix of $X\delta$ such that $\gamma \Rightarrow^* a\omega$, $a$ is the first symbol of $v_j$. 
\end{center}
The way this could be solved is by doing a breadth first search (BFS) on all the possible prefixes to see if the $a$ can be derived. Then just choose the shortest of the prefixes that can derive $a\omega$.  The problem here is you would need a condition for stopping the BFS such that it does not try to derive something that can not appear.

An idea is to check if the first symbol of a given derivation has been visited before, if so then skip the derivation. A problem that may occur here is if $X\delta = YYa$ and $Y \Rightarrow^* \varepsilon$ then since $Y$ is visited on the second derivation then that path is not further explored. This was at some point the used implementation in this project and was a cause of trouble It was later realized step 3. had a mistake, and it should say something like.
\begin{center}
    step 3. (a): $\gamma$ is the shortest prefix of $X\delta$ such that $\gamma \Rightarrow^* v_j\omega$. 
\end{center}
But an easier idea was to create all prefixes of $\gamma$ and compute the $\FIRST_k$ of these prefixes. $\gamma$ is then the shortest prefix where $a \in \FIRST_k(\gamma)$. This implementation should is easy to comprehend and fast because of the use of memoization in the $\FIRST$. Due to this the Algorithm 8 was changed to have a step with.
\begin{center}
    step 3. (a): $\gamma$ is the shortest prefix of $X\delta$ such that $v_j \in \FIRST_k(\gamma)$. 
\end{center}