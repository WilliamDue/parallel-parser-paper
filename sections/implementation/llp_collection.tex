When constructing the LLP collection of items sets Algorithm 8 \cite[13]{Vagner2007} is used. The actual Haskell implementation Algorithm 8 should match the implementation, but some implementation details would be needed to be explained. The first implementation detail to consider is. 
\begin{center}
    step 3. (a): ``$\gamma$ is the shortest prefix of $X\delta$ such that $\gamma \Rightarrow^* a \omega$, $a$ is the first symbol of $v_j$'' \cite{errata:Vagner2007}
\end{center}
The way this can be solved is by doing a breadth first search on all the prefixes to see if the $a$ can be derived. Then just choose the shortest of the prefixes that can derive $a\omega$. This is the first implementation that was used when algorithm 8 was implemented. The current implementation creates all prefixes of $\gamma$ and computes $\FIRST_1$ of these prefixes. $\gamma$ is then the shortest prefix where $a \in \FIRST_1(\gamma)$. This should be a faster implementation because the $\FIRST$ implementation uses memoization. This results in $\FIRST$ can become a dictionary look up instead of a breadth first search.