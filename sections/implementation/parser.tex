When the parser generator has created a table, this table will need to be represented as code Futhark somehow. And a parallel bracket matching algorithm to check the LLP configurations match. These problems will be solved in this section.
\subsubsection{String Packing}
The first problem is how do we represent strings within Futhark. Futhark does not have dynamic arrays and is not able to represent arrays of different lengths in an array. This is because Futhark needs to be able to know how much memory will be allocated. The reason, this is a problem, is the function \lstinline|key_to_config| in the parser acts as a table look up which returns the corresponding LLP configuration. These LLP configuration may result in differently sized pushdown stores or number of productions $(\alpha, \omega, \pi)$.

This is actually not completely true since instead of using the LLP configuration $(\alpha, \omega, \pi)$ as is, the list homomorphisms in algorithm 18 \cite[18]{Vagner2007} which results in the tuples $(RBR(\alpha)LBR(\omega^R), \pi)$ besides for the starting pairs $(\epsilon, \vdash w)$ where $w \in T^*$ which becomes $(LBR(\omega^R), \pi)$. This is done such that step 2. from Algorithm 18 \cite[18]{Vagner2007} can be precomputed.

Since \lstinline|key_to_config| is mapped over an array of $(x,y)$ input pairs this will result in the Futhark compiler not knowing how much memory needs to be allocated. This is also a problem since in the parser generator terminals and nonterminals are strings, so they may have different lengths.

To solve this problem string padding can be used. The way this is done is every terminal string is assigned an index $i \in \{0, 1, \dots, |T| - 1\}$ where $|T| - 1 < 2^{32} - 1$. This is because the largest 32-bit value of $2^{32} - 1$ is used as string padding. When dealing with a sequence of production then the indexes $i \in \{0, 1, \dots, |P| - 1\}$ where $|P| - 1 < 2^{32} - 1$ for the same reason as for terminal strings. When a string is made of nonterminals and terminals then each nonterminal is are assigned an index from $i \in \{|T| - 1, |T|, |N| + |T| - 1\}$ where $2^{32} - 1 \leq i \leq 2^{64} - 1$. This is done since instead of assigning a specific integer as being padding the non-recursive sum types from Futhark can be used. This is a good choice since they are already used to label an element on the pushdown store is a \lstinline|#left| or \lstinline|#right| bracket.

These considerations result in the following table look up function which is created by the parser generator. This function does pattern matching on the given input pair.

\begin{lstlisting}[basicstyle=\ttfamily\scriptsize]
def key_to_config (key : ((u32), (u32))) : maybe ([]bracket, []u32) =
    match key
    case ((0), (1)) -> #just ([#right 5, #right 1], [3])
    case ((0), (2)) -> #just ([#right 5, #left 5], [1])
    case ((0), (3)) -> #just ([#right 5, #epsilon], [2])
    case ((2), (1)) -> #just ([#right 5, #right 1], [3])
    case ((2), (2)) -> #just ([#right 5, #left 5], [1])
    case ((2), (3)) -> #just ([#right 5, #epsilon], [2])
    case ((3), (1)) -> #just ([#right 1, #epsilon], [u32.highest])
    case ((4294967295), (0)) -> #just ([#left 1, #left 5], [0])
    case _ -> #nothing
\end{lstlisting}
The returned configurations are inside a \lstinline|Maybe|-like type. So if at any point \lstinline|#nothing| is returned then an invalid string pair was returned and therefore $(x,y)$ was not an admissable pair. This table function above arises from the following grammar where each subscript correspondx to the integers.
\begin{align*}
    0. \: S'_0 \to \vdash_0 A_1 \dashv_1 \qquad 1. \: A_1 \to a_2 A_1 \qquad 2. \: A_1 \to b_3 \qquad 3. \: A_1 \to \epsilon
\end{align*}

\subsubsection{Bracket Matching}
Besides this the Futhark implementation matches Algorithm 18. A missing piece is the parallel bracket matching which is not described in the LLP paper \cite[text]{Vagner2007}. The implementation used takes a lot of inspiration from the implementation described on the Futhark \cite{futhark:parens}. The differences are a balancing check is made as can be seen below.
\begin{lstlisting}[basicstyle=\ttfamily\scriptsize]
def depths [n] (input : [n]bracket) : maybe ([n]i64) =
    let left_brackets =
      input
      |> map (is_left)
    let bracket_scan =
      left_brackets
      |> map (\b -> if b then 1 else -1)
      |> scan (+) 0
    let result =
      bracket_scan
      |> map2 (\a b -> b - i64.bool a) left_brackets
    in if any (<0) bracket_scan || last bracket_scan != 0
    then #nothing
    else #just result
\end{lstlisting}
This is done using \lstinline|bracket_scan| which must only contain positive values and the last value must be zero. If these predicates are not fulfilled then it is known the brackets are not balanced. The returned result is the \lstinline|bracket_scan| returned such that it is adjusted for the left brackets being one then the right. Here the non-recursive sum-type from Futhark are utilized once again where \lstinline|#nothing| is returned if an error occurs.

Something to note is the glue \cite[7]{Vagner2007} reduce could have been used instead of algorithm 18 \cite[18]{Vagner2007}. This is a bad choice for two reasons, the first is it is slow \cite[17]{Vagner2007}. The second reason is Futhark does not have dynamic arrays and does not allow for using concatenation when using the built-in \lstinline|reduce| function.

Assuming the number of processes used is the number of terminals then the $\LLP(q, k)$ parsing algorithm using bracket matching is $O(\log n)$ where $n$ is the number of terminals in the input string \cite[19]{Vagner2007}.